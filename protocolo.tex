%!TEX encoding = UTF-8 Unicode
%!TEX program = xelatex
%% This template licensed under CC-BY-NC-SA by Koenraad De Smedt
\documentclass[final, fmstyle, 12pt]{article}
\usepackage[utf8]{inputenc}
%\usepackage[spanish]{babel}
\usepackage[margin=24mm]{geometry}
\usepackage{fontspec,xltxtra,polyglossia,titling,graphicx,dingbat}
\usepackage{verbatim,gb4e,synttree,multicol} % choose or add what you need   
\usepackage[colorlinks,urlcolor=blue,citecolor=black,linkcolor=black]{hyperref}
\setmainfont[Mapping=tex-text]{Times New Roman} % or another similar font
\setdefaultlanguage{spanish}
%\setotherlanguages{english}
\usepackage[numbers]{natbib}
\usepackage{xspace}
\usepackage{float}
\usepackage{dirtytalk}
\usepackage{subfiles}
\hypersetup{
    colorlinks=true,
    linkcolor=black,
    filecolor=magenta,      
    urlcolor=black,
    pdftitle={Memoria experiencia profesional},
    pdfpagemode=FullScreen,
    }
\urlstyle{same}
\frenchspacing
%\newcommand{\checkmark}{x}

\title{MEMORIA DE EXPERIENCIA PROFESIONAL EN LA EMPRESA}
\author{Hector Hugo Vidaña Arrieta}
\date{\today}
\hyphenation{unicode}
\usepackage{url}
\usepackage{hyperref}
%\usepackage[bitstream-charter]{mathdesign}
\begin{document}
\renewcommand{\tablename}{Tabla}
	\thispagestyle{empty}
\begin{center} \vfill
{\Large UNIVERSIDAD AUTÓNOMA DE CIUDAD JUÁREZ}\\
\vspace{6mm}
{\large Instituto de Ingeniería y Tecnología\\
\vspace{6mm}
Departamento de Ingeniería Eléctrica y Computación
\vspace{20mm}

\includegraphics [scale=0.7]{imagenes/escudo-uacj} 
\vspace{10mm}


\thetitle\\
\vspace{15mm}

Protocolo de investigación presentado por:\\
\vspace{6mm}
\theauthor\hspace{10mm} 159957\\
\vspace{10mm}
Requisito para la obtención del título de\\
\vspace{6mm}
INGENIERO EN SOFTWARE\\
\vspace{10mm}

Asesor:\\
{Julia Patricia Sanchez Solis}\\
} \vfill
	Ciudad Juárez, Chihuahua \hspace{70mm}\today

% {\small \href{https://creativecommons.org/licenses/by/4.0/}{\includegraphics[height=1.2em]{cc-by}} by \theauthor}
\clearpage

\end{center}


\section{Introducción}
\setlength{\parskip}{1em} 
Este docuent presenta la trayectoria de Héctor Hugo Vidaña Arrieta en el campo de la Ingeniería de Software, resaltando sus logros y contribuciones clave a lo largo de su carrera. Desde una temprana edad, ha demostrado un profundo interés y compromiso con la ingeniería de software, participando activamente en proyectos de diversa índole y envergadura.

Su formación académica en la Universidad Autónoma de Ciudad Juárez (UACJ) le proporcionó una base sólida de conocimientos y habilidades, que supo complementar y enriquecer a través de su experiencia laboral. Vidaña Arrieta ha colaborado en proyectos de relevancia tanto a nivel estatal como nacional, aplicando su expertise para alcanzar resultados exitosos.

A lo largo de su carrera, ha desempeñado roles diversos, desde desarrollador de software y administrador de bases de datos hasta desarrollador de sistemas embebidos y líder de equipos de software. Su capacidad para adaptarse a diferentes entornos y desafíos, así como su compromiso con la calidad y la innovación, le han permitido destacar en cada uno de sus proyectos.

Entre sus contribuciones más notables se encuentran el desarrollo de un sistema financiero para la gestión de proyectos y gastos, un sistema de conteo vehicular basado en inteligencia artificial, un sistema TCP/IP de geolocalización y telemetría, un sistema de administración de fraccionamientos y módulos para un sistema ERP de gran escala.

La experiencia laboral de Vidaña Arrieta no solo ha consolidado sus habilidades técnicas, sino que también ha fortalecido su capacidad para trabajar en equipo, liderar proyectos y comunicarse eficazmente con clientes y stakeholders. Su trayectoria es un testimonio de su pasión por la ingeniería de software y su compromiso con el desarrollo de soluciones innovadoras que generen un impacto positivo en la sociedad.

Este informe respalda sólidamente la solicitud de titulación por Experiencia Laboral de Héctor Hugo Vidaña Arrieta, demostrando que su formación académica, habilidades adquiridas y logros profesionales lo convierten en un ingeniero de software altamente competente y preparado para enfrentar los desafíos del campo. Su trayectoria es un ejemplo inspirador para futuros ingenieros y una muestra del potencial que la experiencia laboral puede aportar al desarrollo profesional en el ámbito de la ingeniería de software.
% [Indique de forma clara y coherente en una cuartilla cuál es la nueva contribución, su importancia y por qué es adecuado para sistemas computacionales. Se sugiere para su redacción seguir los cinco pasos siguientes: 1) Establezca el campo de investigación al que pertenece el proyecto, 2) describa los aspectos del problema más que ya han sido estudiado por otros investigadores, 3) explique el área de oportunidad que pretende cubrir el proyecto propuesto, 4) describa el propósito/objetivo del proyecto y 5) proporcione el valor positivo o la justificación para llevar a cabo el proyecto.] 
% \newpage
\vfill % Intenta evitar el salto de página aquí
% Entorno Laboral
\subfile{entorno-laboral}


\subfile{experiencia-laboral}

\vfill % Intenta evitar el salto de página aquí
\subfile{Resultados}

\subfile{conclusiones}

\newpage
\bibliographystyle{IEEEtranN}
% \bibliographystyle{plainurl}
\bibliography{referencias}
%\nocite{*} 

\clearpage
\appendix

\section{Apéndice}

\end{document}  