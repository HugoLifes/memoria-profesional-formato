%!TEX encoding = UTF-8 Unicode
%!TEX program = xelatex
%% This template licensed under CC-BY-NC-SA by Koenraad De Smedt
\documentclass[final, fmstyle, 12pt]{article}
\usepackage[utf8]{inputenc}
%\usepackage[spanish]{babel}
\usepackage[margin=24mm]{geometry}
\usepackage{fontspec,xltxtra,polyglossia,titling,graphicx,dingbat}
\usepackage{verbatim,gb4e,synttree,multicol} % choose or add what you need   
\usepackage[colorlinks,urlcolor=blue,citecolor=black,linkcolor=black]{hyperref}
\setmainfont[Mapping=tex-text]{Times New Roman} % or another similar font
\setdefaultlanguage{spanish}
%\setotherlanguages{english}
\usepackage[numbers]{natbib}
\usepackage{xspace}
\usepackage{float}
\usepackage{dirtytalk}
\usepackage{subfiles}
\hypersetup{
    colorlinks=true,
    linkcolor=black,
    filecolor=magenta,      
    urlcolor=black,
    pdftitle={Memoria experiencia profesional},
    pdfpagemode=FullScreen,
    }
\urlstyle{same}
\frenchspacing
%\newcommand{\checkmark}{x}

\title{MEMORIA DE EXPERIENCIA PROFESIONAL GRUPO TOMZA}
\author{Héctor Hugo Vidaña Arrieta}
\date{\today}
\hyphenation{unicode}
\usepackage{url}
\usepackage{hyperref}
%\usepackage[bitstream-charter]{mathdesign}
\begin{document}
\renewcommand{\tablename}{Tabla}
	\thispagestyle{empty}
\begin{center} \vfill
{\Large UNIVERSIDAD AUTÓNOMA DE CIUDAD JUÁREZ}\\
\vspace{6mm}
{\large Instituto de Ingeniería y Tecnología\\
\vspace{6mm}
Departamento de Ingeniería Eléctrica y Computación
\vspace{20mm}

\includegraphics [scale=0.7]{Imagenes/escudo-uacj} 
\vspace{10mm}


\thetitle\\
\vspace{15mm}

Protocolo de investigación presentado por:\\
\vspace{6mm}
\theauthor\hspace{10mm} 159957\\
\vspace{10mm}
Requisito para la obtención del título de\\
\vspace{6mm}
INGENIERO EN SOFTWARE\\
\vspace{10mm}

Asesora:\\
{Dra. Julia Patricia Sánchez Solis}\\
} \vfill
	Ciudad Juárez, Chihuahua \hspace{70mm}\today

% {\small \href{https://creativecommons.org/licenses/by/4.0/}{\includegraphics[height=1.2em]{cc-by}} by \theauthor}
\clearpage

\end{center}


\section{Introducción}
\setlength{\parskip}{1em} 
El presente documento expone las experiencias profesionales de Héctor Hugo Vidaña Arrieta, 
adquiridas durante su trayectoria laboral en tres empresas del sector tecnológico: 
Grupo Tomza, Apia Ingeniería y Soluciones Móviles y Comunicaciones. 
Empresas que se desenvuelven en el dinámico sector de las tecnologías de la información, 
con un enfoque particular en el desarrollo de software, administración de bases de datos,
sistemas embebidos e inteligencia artificial. Grupo Tomza y Soluciones Móviles se caracteriza por ser empresas consolidadas con una amplia trayectoria en el mercado tecnológico, mientras que Apia Ingeniería es una startup en crecimiento que se destaca por su cultura ágil y flexible.

En este documento se detallarán los proyectos en los que Héctor Hugo Vidaña participó, destacando su relación con las áreas de conocimiento de la ingeniería de software, tales como  desarrollo de software, administración de bases de datos,
sistemas embebidos e inteligencia artificial.  Cada proyecto se describirá a detalle, incluyendo el contexto, los objetivos, las metodologías empleadas y los resultados obtenidos.

A través de la descripción de estas experiencias, se busca ilustrar la aplicación práctica de la formación académica de Héctor Hugo Vidaña Arrieta y cómo ésta contribuyó al desarrollo de su perfil profesional. Se  evidenciará cómo los conocimientos teóricos adquiridos durante sus estudios se complementaron con la experiencia práctica en el campo laboral,  permitiéndole afrontar los desafíos y  contribuir al éxito de los proyectos en los que se involucró.

Finalmente, se presentarán las conclusiones derivadas de estas experiencias, incluyendo una reflexión sobre el impacto de la formación académica en el desempeño profesional,  así como  sugerencias para fortalecer el programa de estudios de la ingeniería de software.
% [Indique de forma clara y coherente en una cuartilla cuál es la nueva contribución, su importancia y por qué es adecuado para sistemas computacionales. Se sugiere para su redacción seguir los cinco pasos siguientes: 1) Establezca el campo de investigación al que pertenece el proyecto, 2) describa los aspectos del problema más que ya han sido estudiado por otros investigadores, 3) explique el área de oportunidad que pretende cubrir el proyecto propuesto, 4) describa el propósito/objetivo del proyecto y 5) proporcione el valor positivo o la justificación para llevar a cabo el proyecto.] 
% \newpage
\newpage
\tableofcontents 

% Entorno Laboral
\newpage 
\subfile{entorno-laboral}

\newpage 
\subfile{experiencia-laboral}

\newpage 
\subfile{Resultados}
\newpage 
\subfile{conclusiones}

\newpage
\bibliographystyle{IEEEtranN}
% \bibliographystyle{plainurl}
\begin{thebibliography}{99}
    \bibitem{obrien2013} O'Brien, J. A., y Marakas, G. M. (2013). \textit{Management Information Systems}. McGraw-Hill Irwin.
    \bibitem{turban2010} Turban, E., Sharda, R., Delen, D., y King, D. (2010). \textit{Business intelligence: A managerial approach}. Pearson Education.
    \bibitem{russell2010} Russell, S. J., y Norvig, P. (2010). \textit{Artificial Intelligence: A Modern Approach}. Pearson Education.
\end{thebibliography}
%\nocite{*} 






\end{document}  