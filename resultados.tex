\documentclass[protocolo.tex]{subfiles}

\begin{document}

\section{Resultados}
Todo proyecto realizado es operacional y se encuentra actualmente en funcionamiento. La retroalimentación positiva recibida por parte de sus superiores  confirma la calidad de su trabajo como Ingeniero en Software. Ha manejado con éxito la calidad, gestión, diseño y desarrollo en cada uno de los proyectos en los que ha trabajado, dejando una huella positiva en las empresas donde ha colaborado.\vspace{4mm}

Como próximo ingeniero en software, reconoce que aún queda mucho por aportar y crecer.  Si bien sus errores han sido pocos,  es consciente de que, como principiante,  cometió algunos.  Sin embargo,  los considera parte del proceso de aprendizaje en este campo laboral y  nunca permitió que afectaran al desarrollo de los proyectos o a su entorno laboral.\vspace{4mm}

Una aportación principal dentro de su experiencia ha sido la inteligencia emocional, la cual le ha permitido  relacionarse mejor con sus compañeros de trabajo, superiores e incluso con personas ajenas a la empresa, como clientes o personal en capacitación.  Reconoce la importancia de  manejar las diferentes  posibilidades que puedan surgir dentro de un gran proyecto de desarrollo de software.

\subsection{Resultados del Proyecto DataFire}

El proyecto \textit{DataFire} consistió en el desarrollo de un sistema financiero empresarial para la gestión de costos, pago de nóminas y presupuestos de proyectos. Este sistema se implementó con el objetivo de optimizar la administración del dinero de la compañía y mejorar el control sobre las actividades dentro de los proyectos.\vspace{4mm}

Como resultados de la implementación de \textit{DataFire}, se logró:

\begin{itemize}
\item Reducir los gastos de la empresa en un 50\% gracias a la automatización de procesos y la optimización de recursos.
\item Mejorar la detección del rendimiento de los empleados, lo que permitió identificar áreas de oportunidad y tomar decisiones más informadas sobre la asignación de recursos humanos.
\end{itemize}

El proyecto \textit{DataFire} le permitió adquirir experiencia en el desarrollo de sistemas financieros empresariales y fortalecer mis habilidades en el análisis de datos, la gestión de bases de datos y la programación en \textit{Flutter} y \textit{NodeJs}.


\subsection{Resultados del Proyecto VehicleTracking}

El proyecto  \textit{VehicleTracking} se centró en el desarrollo de un sistema de conteo de vehículos con inteligencia artificial (IA) para el aforo vehicular.  Este sistema, aún en versión de pruebas, se implementó con el objetivo de automatizar el conteo de vehículos y reducir la necesidad de intervención humana en esta tarea.\vspace{4mm}

Como resultado de las pruebas de \textit{VehicleTracking}, se observó:

\begin{itemize}
\item Un aumento del 30\% en la eficiencia del conteo vehicular,  logrado gracias a la automatización del proceso mediante el uso de IA.
\end{itemize}

El proyecto \textit{VehicleTracking} le permitió adquirir experiencia en el desarrollo de sistemas de visión artificial y aprendizaje automático, así como en el procesamiento de imágenes y la programación en  \textit{Python},  \textit{YOLO} y  \textit{Ultralytics}.  Además, le brindó la oportunidad de trabajar con tecnologías de vanguardia y contribuir a la automatización de procesos en el ámbito del aforo vehicular.

\subsection{Resultados del Proyecto CRM}

El proyecto CRM consistió en el desarrollo de un oyente basado en conexiones TCP/IP, que procesara datos en hexadecimal que contiene información telemetrica y geoespacial para posteriormente ingresarlos en la nube y poder visualizar los datos de manera constante y eficiente. \vspace{4mm}

Como resultado del desarrollo del CRM, se logró:

\begin{itemize}
    \item Una eficiencia del 80\% al cambiar de lenguaje de programación esto ayudo a que el sistema fuera mas rápido, entendible y escalable.
\end{itemize}

Estos resultados ayudaron a que la empresa ofreciera un mejor servicio de \textit{tracking} para las empresas que gozaban de estos beneficios al contar con Calamp integrado en sus unidades. \vspace{4mm}

El sistema CRM le permitió adquirir conocimientos para realizar un oyente para posteriormente procesar y manejar datos en hexadecimal además de fortalecer las habilidades de Héctor Hugo en lenguaje \textit{Python} y crecer en el ámbito de las conexiones TCP/IP.

\subsection{Resultados del Proyecto FollowMe}

El proyecto \textit{FollowMe} consistió en el desarrollo de un sistema para visualizar tráileres con cargas importantes mediante \textit{GoogleMaps}.  Se utilizó un dispositivo llamado Calamp que emitía señales en hexadecimal, para lo cual se desarrolló un oyente llamado CRM que capturaba la información y  actualizaba la base de datos con la ubicación en tiempo real.  Además, se implementó una vista para visualizar la ubicación y la telemetría del tráiler, con el fin de capturar datos y tomar medidas en tiempo real en caso de cualquier circunstancia.\vspace{4mm}

Como resultado de la implementación de \textit{Followme}, se logró:

\begin{itemize}
\item Una mejora del 80\% en la eficiencia del seguimiento de tráileres, ya que la información se  presentó  de forma más accesible y organizada para el personal encargado.
\end{itemize}

El proyecto \textit{Followme} le permitió adquirir experiencia en el desarrollo de sistemas de rastreo en tiempo real y la integración con dispositivos de hardware.  También le brindó la oportunidad de aplicar mis habilidades en la gestión de bases de datos y el desarrollo de interfaces de usuario.


\subsection{Resultados del Proyecto AdCom}

El proyecto AdCom se enfocó en el desarrollo de un sistema de administración de fraccionamientos con el objetivo de  mejorar la comunicación entre la administración y los residentes,  facilitar el acceso a la información y agilizar los procesos administrativos.  A través de este sistema, los residentes pueden  familiarizarse con su residencia,  disfrutar de sus recursos,  consultar sus deudas y realizar pagos de forma online.\vspace{4mm}

Como resultados de la implementación de Adcom, se logró:

\begin{itemize}
\item Una reducción del 90\% en las llamadas al personal de asistencia,  gracias a la disponibilidad de información y la  facilidad  de  uso  del  sistema.
\item Una mejora del 50\% en los reportes de incidentes dentro del fraccionamiento,  al  proporcionar  un  canal  de  comunicación  más  eficiente  y  accesible  para  los  residentes.
\item Una reducción del 100\% en el tiempo dedicado a la gestión de pagos,  al  implementar  un  sistema  de  pagos  vía  móvil.
\end{itemize}

Estos resultados contribuyeron a que la empresa  optimizara la administración de su personal, liberando tiempo para dedicarlo a otras tareas que requerían  mayor  atención.\vspace{4mm}

El proyecto Adcom le permitió adquirir experiencia en el desarrollo de sistemas web para la gestión de comunidades,  la integración de pasarelas de pago y la  implementación  de  sistemas  de  notificación.  Además,  fortaleció  mis  habilidades  en  el  diseño  de  interfaces  de  usuario  intuitivas  y  la  gestión  de  proyectos  con  enfoque  en  la  experiencia  del  usuario.

\subsection{Resultados del Proyecto Zae}

El proyecto Zae consistió en el desarrollo e implementación de un sistema ERP para la gestión integral de las empresas del Grupo Tomza. Este sistema se diseñó para administrar las ventas, empleados, facturas, encargos, llamadas y reportes de las plantas en los diferentes estados de México.\vspace{4mm}

Como resultado de la implementación de Zae, se logró:

\begin{itemize}
\item Una mejora del 90\% en la visualización de datos,  facilitando el acceso a la información relevante para la toma de decisiones.
\item Un aumento del 70\% en la eficiencia de la generación de reportes,  agilizando los procesos administrativos y operativos.
\item Un incremento del 80\%o en las ventas de gas,  gracias a la mejora en la calidad, rendimiento y usabilidad del sistema.
\end{itemize}

Zae se convirtió en el sistema clave de la empresa,  permitiendo una mejor gestión de las ventas y una visión más completa del estado general de la compañía.\vspace{4mm}

El proyecto Zae le permitió adquirir experiencia en el desarrollo e implementación de sistemas ERP,  la integración de diferentes módulos y la gestión de bases de datos a gran escala.  Además,  fortaleció las  habilidades de Héctor Hugo Vidaña Arrieta en el análisis de requerimientos,  el diseño de soluciones  y  el  trabajo  en  equipo  con  diferentes  áreas  de  la  empresa.\vspace{4mm}


\subsection{Resultados del Proyecto Zae Ejecutivo}

El proyecto Zae Ejecutivo se enfocó en el desarrollo de un sistema web que globaliza los datos del sistema Zae y los presenta de forma visual con gráficos  y  datos  concretos.  Este sistema se diseñó para  facilitar  la  comprensión  de  métricas  clave  como  las  ventas,  el  rendimiento  de  presupuestos  y  el  rendimiento  de  las  plantas,  y  permitir  la  creación  de  reportes  globales.  Zae  Ejecutivo  se  orientó  a  gerentes  y  altos  ejecutivos  de  la  compañía.\vspace{4mm}

Como resultado de la implementación de Zae Ejecutivo, se logró:

\begin{itemize}
\item Una mejora del 85\%  en la visualización y el entendimiento de los datos por parte de los encargados de nivel mayor,  facilitando la toma de decisiones estratégicas.
\item Un aumento del 70\%  en la usabilidad y el rendimiento del sistema,  garantizando la disponibilidad y estabilidad de la información.
\end{itemize}

Zae Ejecutivo  contribuyó a que los ejecutivos de la empresa administraran mejor sus recursos y  planificaran  sus  próximos  movimientos  con  base  en  información  confiable  y  actualizada.\vspace{4mm}

El proyecto Zae Ejecutivo le permitió adquirir experiencia en el desarrollo de sistemas de  \textit{business intelligence},  la visualización de datos y la creación de  \textit{Dashboards}  interactivos.  Además,  fortaleció las habilidades de Héctor Hugo Vidaña Arrieta en  el  diseño  de  interfaces  de  usuario  para  altos  ejecutivos  y  la  presentación  de  información  de  forma  clara  y  concisa.

\end{document}