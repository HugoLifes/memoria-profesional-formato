\documentclass[protocolo.tex]{subfiles}

\begin{document}

\section{Resultados}
Cada proyecto que he realizado es operacional y actualmente está en funcionamiento.
Además, recibir buena retroalimentación por parte de mis superiores dentro del entorno
laboral me ha hecho saber que estoy haciendo un buen trabajo como Ingeniero en Software.
He manejado calidad, gestión, diseño y desarrollo para cada uno de los proyectos en los que
he trabajado, permitiendo dejar una buena huella en las empresas en las que he colaborado.
Como próximo ingeniero en software, aún queda mucho por aportar y crecer.
Mis errores han sido pocos, pero como principiante que fui, llegué a cometer algunos.
Aunque es parte de crecer en este campo laboral, jamás dejé que afectarán al desarrollo o a
mi entorno laboral, ya que forman parte de mi preparación como Ingeniero en Software.
Una aportación principal dentro de mi experiencia ha sido la inteligencia emocional, ya que
me ha permitido relacionarme mejor con mis compañeros de trabajo, superiores o incluso
con personas ajenas, como puede ser un cliente o el personal que está por ser capacitado.
Todo esto coexiste en el campo laboral de la Ingeniería de Software y hay que saber
manejar estas posibilidades que puedan surgir dentro de un gran proyecto de desarrollo.

\subsection{Resultados del Proyecto Datafire}

El proyecto Datafire consistió en el desarrollo de un sistema financiero empresarial para la gestión de costos, pago de nóminas y presupuestos de proyectos. Este sistema se implementó con el objetivo de optimizar la administración del dinero de la compañía y mejorar el control sobre las actividades dentro de los proyectos.

Como resultado de la implementación de Datafire, se logró:

\begin{itemize}
\item Reducir los gastos de la empresa en un 50 porciento gracias a la automatización de procesos y la optimización de recursos.
\item Mejorar la detección del rendimiento de los empleados, lo que permitió identificar áreas de oportunidad y tomar decisiones más informadas sobre la asignación de recursos humanos.
\end{itemize}

El proyecto Datafire me permitió adquirir experiencia en el desarrollo de sistemas financieros empresariales y fortalecer mis habilidades en el análisis de datos, la gestión de bases de datos y la programación en flutter y nodejs.


\subsection{Resultados del Proyecto I Tracking}

El proyecto I Tracking se centró en el desarrollo de un sistema de conteo de vehículos con inteligencia artificial (IA) para el aforo vehicular.  Este sistema, aún en versión de pruebas, se implementó con el objetivo de automatizar el conteo de vehículos y reducir la necesidad de intervención humana en esta tarea.

Como resultado de las pruebas de I Tracking, se observó:

\begin{itemize}
\item Un aumento del 30 porciento en la eficiencia del conteo vehicular,  logrado gracias a la automatización del proceso mediante el uso de IA.
\end{itemize}

El proyecto I Tracking me permitió adquirir experiencia en el desarrollo de sistemas de visión artificial y aprendizaje automático, así como en el procesamiento de imágenes y la programación en python, Yolo y ultralytics.  Además, me brindó la oportunidad de trabajar con tecnologías de vanguardia y contribuir a la automatización de procesos en el ámbito del aforo vehicular.

\subsection{Resultados del Proyecto Followme}

El proyecto Followme consistió en el desarrollo de un sistema para rastrear tráileres con cargas importantes mediante una conexión TCP/IP.  Se utilizó un dispositivo llamado Calamp que emitía señales en hexadecimal, para lo cual se desarrolló un oyente que capturaba la información y  actualizaba la base de datos con la ubicación en tiempo real.  Además, se implementó una vista para visualizar la ubicación y la telemetría del tráiler, con el fin de capturar datos y tomar medidas en tiempo real en caso de cualquier circunstancia.

Como resultado de la implementación de Followme, se logró:

\begin{itemize}
\item Una mejora del 80 porciento en la eficiencia del seguimiento de tráileres, ya que la información se  presentó  de forma más accesible y organizada para el personal encargado.
\end{itemize}

El proyecto Followme me permitió adquirir experiencia en el desarrollo de sistemas de rastreo en tiempo real, el manejo de protocolos de comunicación TCP/IP, la interpretación de datos hexadecimales y la integración con dispositivos de hardware.  También me brindó la oportunidad de aplicar mis habilidades en la gestión de bases de datos y el desarrollo de interfaces de usuario.


\subsection{Resultados del Proyecto Adcom}

El proyecto Adcom se enfocó en el desarrollo de un sistema de administración de fraccionamientos con el objetivo de  mejorar la comunicación entre la administración y los residentes,  facilitar el acceso a la información y agilizar los procesos administrativos.  A través de este sistema, los residentes pueden  familiarizarse con su residencia,  disfrutar de sus recursos,  consultar sus deudas y realizar pagos de forma online.

Como resultado de la implementación de Adcom, se logró:

\begin{itemize}
\item Una reducción del 90 porciento en las llamadas al personal de asistencia,  gracias a la disponibilidad de información y la  facilidad  de  uso  del  sistema.
\item Una mejora del 50 porciento en los reportes de incidentes dentro del fraccionamiento,  al  proporcionar  un  canal  de  comunicación  más  eficiente  y  accesible  para  los  residentes.
\item Una reducción del 100 porciento en el tiempo dedicado a la gestión de pagos,  al  implementar  un  sistema  de  pagos  vía  móvil.
\end{itemize}

Estos resultados contribuyeron a que la empresa  optimizara la administración de su personal,  liberando tiempo para dedicarlo a otras tareas que requerían  mayor  atención.

El proyecto Adcom me permitió adquirir experiencia en el desarrollo de sistemas web para la gestión de comunidades,  la integración de pasarelas de pago y la  implementación  de  sistemas  de  notificación.  Además,  fortaleció  mis  habilidades  en  el  diseño  de  interfaces  de  usuario  intuitivas  y  la  gestión  de  proyectos  con  enfoque  en  la  experiencia  del  usuario.

\subsection{Resultados del Proyecto Zae}

El proyecto Zae consistió en el desarrollo e implementación de un sistema ERP para la gestión integral de las empresas del Grupo Tomza. Este sistema se diseñó para administrar las ventas, empleados, facturas, encargos, llamadas y reportes de las plantas en los diferentes estados de México.

Como resultado de la implementación de Zae, se logró:

\begin{itemize}
\item Una mejora del 90 porciento en la visualización de datos,  facilitando el acceso a la información relevante para la toma de decisiones.
\item Un aumento del 70 porciento en la eficiencia de la generación de reportes,  agilizando los procesos administrativos y operativos.
\item Un incremento del 80 porciento en las ventas de gas,  gracias a la mejora en la calidad, rendimiento y usabilidad del sistema.
\end{itemize}

Zae se convirtió en el sistema clave de la empresa,  permitiendo una mejor gestión de las ventas y una visión más completa del estado general de la compañía.

El proyecto Zae me permitió adquirir experiencia en el desarrollo e implementación de sistemas ERP,  la integración de diferentes módulos y la gestión de bases de datos a gran escala.  Además,  fortaleció mis habilidades en el análisis de requerimientos,  el diseño de soluciones  y  el  trabajo  en  equipo  con  diferentes  áreas  de  la  empresa.


\subsection{Resultados del Proyecto Zae Ejecutivo}

El proyecto Zae Ejecutivo se enfocó en el desarrollo de un sistema web que globaliza los datos del sistema Zae y los presenta de forma visual con gráficos  y  datos  concretos.  Este sistema se diseñó para  facilitar  la  comprensión  de  métricas  clave  como  las  ventas,  el  rendimiento  de  presupuestos  y  el  rendimiento  de  las  plantas,  y  permitir  la  creación  de  reportes  globales.  Zae  Ejecutivo  se  orientó  a  gerentes  y  altos  ejecutivos  de  la  compañía.

Como resultado de la implementación de Zae Ejecutivo, se logró:

\begin{itemize}
\item Una mejora del 85 porciento en la visualización y el entendimiento de los datos por parte de los encargados de nivel mayor,  facilitando la toma de decisiones estratégicas.
\item Un aumento del 70 porciento en la usabilidad y el rendimiento del sistema,  garantizando la disponibilidad y estabilidad de la información.
\end{itemize}

Zae Ejecutivo  contribuyó a que los ejecutivos de la empresa administraran mejor sus recursos y  planificaran  sus  próximos  movimientos  con  base  en  información  confiable  y  actualizada.

El proyecto Zae Ejecutivo me permitió adquirir experiencia en el desarrollo de sistemas de  \textit{business intelligence},  la visualización de datos y la creación de  \textit{dashboards}  interactivos.  Además,  fortaleció  mis  habilidades  en  el  diseño  de  interfaces  de  usuario  para  altos  ejecutivos  y  la  presentación  de  información  de  forma  clara  y  concisa.

\end{document}