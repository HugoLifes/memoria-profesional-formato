\documentclass[protocolo.tex]{subfiles}
\begin{document}

\section{Conclusiones}
Para terminar mi recorrido en el campo del desarrollo de software ha sido realmente satisfactoria y enriquecedora tanto a nivel personal como profesionalmente. En este camino he tenido la maravillosa oportunidad de aportar de forma positiva a la sociedad a través de mis creaciones lo cual me llena de un gran orgullo. Sin duda alguna he enfrentado diversos desafíos y cometido errores en el camino pero he sabido convertir esas dificultades en oportunidades para aprender y crecer La experiencia de instalar software en entornos desconocidos u otros contextos ha sido emocionante y me ha llevado hacia logros significativos en mi vida profesional hasta ahora. Además de esto he aprendido cómo mantenerme actualizado ante la evolución constante de la tecnología y continúan las lecciones diarias que demuestran que ser un aprendiz comprometido es fundamental para crecer como ingeniero de software eficiente.
Estoy muy agradecido por todo lo que mis profesores me han enseñado.

Los desafíos que enfrentaré en el futuro me han fortalecido en mi deseo de ser ingeniero.
un software competente y dedicado.

Durante mi formación en ingeniería de software adquirí las habilidades necesarias para llevar a cabo las responsabilidades en los proyectos mencionados sin requerir una capacitación previa específica. Esto evidencia la efectividad y pertinencia del programa académico que cursé.

Al finalizar la carrera y tras las experiencias laborales descritas, se puede afirmar que se  adquirieron los conocimientos necesarios que marca el perfil de egreso de la Ingeniería de Software. En particular, 
se desarrollaron habilidades en análisis y diseño de sistemas, programación, gestión de proyectos, trabajo en equipo y enseñanza. Sin embargo, se considera que el programa de estudios podría fortalecerse en nuevas tecnologías como creación de modelos de lenguaje o modelos de visión ya que requieren un amplio nivel de conocimientos para generar nuevos modelos de inteligencia artificial (IA).  Estas mejoras contribuirían a que los futuros egresados cuenten con una formación aún más completa y se adapten de mejor manera a las demandas del mercado laboral.
\end{document}