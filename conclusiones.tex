\documentclass[protocolo.tex]{subfiles}
\begin{document}

\section{Conclusiones}
El recorrido en el campo del desarrollo de software ha sido realmente
satisfactorio y enriquecedora tanto a nivel personal como profesionalmente. 
En este camino se ha tenido la maravillosa oportunidad de aportar de forma 
positiva a la sociedad a través de las creaciones realizadas lo cual ha sido un logro.
Tambien se han enfrentado diversos desafíos y cometido errores en el camino pero se ha sabido convertir esas dificultades
en oportunidades para aprender y crecer. La experiencia de crear software en entornos desconocidos u otros contextos ha sido 
emocionante y sobre todo un viaje hacia logros significativos en lo profesional y de vida hasta ahora. Además de esto se ha aprendido cómo mantenerse actualizado ante la evolución constante de la tecnología y continúan las lecciones diarias que demuestran que ser un aprendiz comprometido es fundamental para crecer como ingeniero de software eficiente.
Estoy muy agradecido por todo lo que mis profesores me han enseñado.

Los desafíos que se enfrentarán en el futuro han fortalecido el deseo de ser un ingeniero de software competente y dedicado.

Durante la formación en ingeniería de software se adquirieron las habilidades necesarias para llevar a cabo las responsabilidades en los proyectos mencionados sin requerir una capacitación previa específica. Esto evidencia la efectividad y pertinencia del programa académico cursado.

Al finalizar la carrera y tras las experiencias laborales descritas, se puede afirmar que se  adquirieron los conocimientos necesarios que marca el perfil de egreso de la Ingeniería de Software. En particular, 
se desarrollaron habilidades en análisis y diseño de sistemas, programación, gestión de proyectos, trabajo en equipo y enseñanza. Sin embargo, se considera que el programa de estudios podría fortalecerse en nuevas tecnologías como creación de modelos de lenguaje o modelos de visión ya que requieren un amplio nivel de conocimientos para generar nuevos modelos de inteligencia artificial.  Estas mejoras contribuirían a que los futuros egresados cuenten con una formación aún más completa y se adapten de mejor manera a las demandas del mercado laboral.
\end{document}