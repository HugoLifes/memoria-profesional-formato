\documentclass[protocolo.tex]{subfiles}
\begin{document}

\section{Conclusiones}
En conclusión, mi recorrido en el desarrollo de software ha sido una experiencia
profundamente enriquecedora, tanto a nivel personal como profesional. He tenido el
privilegio de contribuir positivamente a la sociedad a través de mis creaciones, lo cual me
llena de una gran satisfacción. A pesar de los desafíos y errores que he encontrado en el
camino, estos se han convertido en valiosas oportunidades de aprendizaje y crecimiento.
La implementación de software en diversos contextos, incluyendo lugares desconocidos, ha
sido un reto estimulante que me ha llevado a alcanzar logros significativos. Además, he
aprendido a adaptarme a la constante evolución de la tecnología, reconociendo que el
aprendizaje continuo es fundamental para ser un ingeniero en software exitoso.
Estoy profundamente agradecido por las enseñanzas de mis profesores, que me han
brindado las herramientas para participar en proyectos de gran envergadura, tanto a nivel
nacional como estatal y empresarial. Estas experiencias me han preparado para enfrentar
los retos del futuro y me han consolidado en mi aspiración de convertirme en un ingeniero
en software competente y comprometido.

Los conocimientos y habilidades adquiridos durante la formación académica en la ingeniería en software resultaron suficientes para desempeñar las funciones y responsabilidades en los proyectos mencionados.  No fue necesaria una capacitación previa específica, lo que demuestra la solidez y pertinencia del programa de estudios cursado.

Al finalizar la carrera y tras las experiencias laborales descritas, se puede afirmar que se  adquirieron los conocimientos necesarios que marca el perfil de egreso de la Ingeniería de Software. En particular, 
se desarrollaron habilidades en análisis y diseño de sistemas, programación, gestión de proyectos, trabajo en equipo y enseñanza .

Sin embargo, se considera que el programa de estudios podría fortalecerse en nuevas tecnologias como creacion de modelos de lenguaje o modelos de vision ya que requieren un amplio nivel de conocimientos para generar nuevos modelos de inteligencia artificial (IA).  Estas mejoras contribuirían a que los futuros egresados cuenten con una formación aún más completa y  se  adapten  de  mejor  manera  a  las  demandas  del  mercado  laboral.
\end{document}