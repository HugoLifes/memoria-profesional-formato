\documentclass[protocolo.tex]{subfiles}
\begin{document}
\vfill % Intenta evitar el salto de página aquí
\section{Experiencia laboral}

\subsection{APIA Ingeniería}

\textbf{Puesto desempeñado:} 
\begin{itemize}
\item Desarrollador de Software
\item Administrador de Base de Datos
\end{itemize}

\textbf{Fechas:}
Desempeñé mis labores desde 3 de febrero de 2018 hasta el 20 de noviembre del 2019 para después volver a realizar labores el 25 de marzo del 2024.

\textbf{Funciones:}
Realizar y administrar los proyectos de software que se llevan a cabo en la empresa, por otra parte también se administra una base de datos que provee información a la empresa y a los sistemas que se realizan.

\textbf{Propuesta:}
\begin{enumerate}
\item Sistema Financiero, que gestiona las ganancias, trabajadores, proyectos y gastos que incurren en la empresa con el fin de gestionar las inversiones que se van realizando en diferentes proyectos y ver representada la información en gráficas.
\item Sistema que cuenta vehículos por video con ayuda de IA, con el fin de eficientizar los aforos vehiculares que se realizan a la hora de hacer estudios de impacto vial.
\end{enumerate}

\textbf{Implementación de propuestas}
\begin{enumerate}
\item Se usó la arquitectura MVC para modularizar y separar la lógica de negocio de la interfaz y experiencia de usuario con el fin de modularizar y eficientizar la programación, también, se usó un framework que acelera el desarrollo de aplicaciones llamado Flutter en conjunto de un modelo de datos relacional administrado haciendo uso de Postgres y gestionando con NodeJs para la creación de las diferentes API’s que el sistema necesita.
\item Sistema en desarrollo en lenguaje Python que realiza aforos vehiculares por medio de video que usa modelos de detección como YOLOx8 para la detección y ultralytics para la segmentación, seguimiento interpolado y el seguimiento visual para el conteo, aún se encuentra en desarrollo.
\end{enumerate}

\textbf{Impacto}
\begin{enumerate}
\item Actualmente el sistema es usado para gestionar proyectos y gastos y tiene un buen funcionamiento, actualmente solo está disponible en ordenadores Windows y se está trabajando para que llegue a más plataformas.
\item El sistema aún se encuentra en desarrollo debido a la complejidad del mismo pero se han realizado pruebas con aforos vehiculares no direccionales, lo cual presentó un visto bueno por parte de la empresa para seguir con el proyecto.
\end{enumerate}

\textbf{Lo aprendido}
\begin{enumerate}
\item Aprendí más acerca de los algoritmos financieros y detecciones de tendencia con el fin de mejorar el impacto de la aplicación a la hora de generar datos más verídicos, por otra parte mejoré en el manejo de la base de datos y realizar funciones como Stored Procedures para optimizar la llamada de datos.
\end{enumerate}

\vfill % Intenta evitar el salto de página aquí 
\subsection{Soluciones Móviles y Comunicaciones S.A DE C.V}

\textbf{Puesto desempeñado:} 
\begin{itemize}
\item Desarrollador de Software FullStack
\item Desarrollador de Sistemas Embebidos.
\end{itemize}

\textbf{Fechas:}
Desempeñé mis labores desde el 18 de marzo del 2021 hasta el 26 de febrero del 2022.

\textbf{Funciones:}
Realizar los proyectos de software que la empresa requirió, como el desarrollo completo de
un sistema o la mejora de un algoritmo en sistemas que la empresa ya tiene y así mismo
manejar el trato directo con los asociados de los diferentes proyectos de software que se
están realizando, como manejo de juntas o nuevas requisiciones del software.


\textbf{Propuesta:}
\begin{enumerate}
\item Sistema TCP/IP de geolocalización y telemetría para camiones y trailers con cargas
importantes con el fin de asegurar el contenido o los diferentes estados que entrega el trailer
o conductor.
\item Sistema de trackeo de paquetes usando el sistema de CALAMP para localizar cargas
dentro de la zona.
\item Sistema que administra fraccionamientos en el estado de Chihuahua, con el fin de
facilitar los accesos del fraccionamiento a los habitantes del mismo desde abrir la puerta
principal, apartar amenidades, ver las deudas dentro del fraccionamiento y también pagar
recibos de luz, agua y gas, reportar fallos dentro del fraccionamiento y más.

\end{enumerate}

\textbf{Implementación de propuestas}
\begin{enumerate}
\item Se desarrolló un oyente en lenguaje Python que recibe un ACK en hexadecimal como
chunk data y hacer un Handshake para responder un ACK para después recibir información
geoespacial y telemétrica para después subir la información a la nube

\item Se desarrolló una interfaz que visualiza paquetes dentro del rango en el que te
encuentras a partir del oyente realizado para recibir información usando tecnología de
Google Maps y Flutter usando una arquitectura Cliente/Servidor.

\item Se desarrolló un sistema que administra fraccionamientos en el estado de Chihuahua,
Chih haciendo uso de Clean Architecture para desarrollos de larga duración y modulares
dentro del Framework de Flutter con el fin de que más programadores pudieran
implementar más módulos y mejorar el entendimiento de código, se usó un diseño propio
para la aplicación que prioriza la originalidad y usabilidad para los diferentes usuarios y
trato directo con los asociados del proyecto para retroalimentación o más requisiciones
\end{enumerate}

\textbf{Impacto}
\begin{enumerate}
\item Actualmente el oyente es utilizado principalmente para alimentar a la base de datos de
ubicaciones para actualizar el mapa que se encuentra de manera online a los que cuentan
con el servicio de Localización y Telemetría tiene un impacto importante ya que es el
algoritmo principal de la empresa.

\item El sistema se encuentra en uso personal por parte de la empresa y ofrece sus servicios a
quienes tienen sistemas Calamp previamente instalados en trailers o camiones, tiene un
impacto medio ya que no todos pueden usar este servicio.

\item Actualmente la aplicación se encuentra en uso y se puede encontrar en tiendas Play
Store y App Store, tiene buen funcionamiento e impacto en las comunidades de
fraccionamientos en Chihuahua.

\end{enumerate}

\textbf{Lo aprendido}
\begin{enumerate}
\item Aprendí a usar mejor las conexiones TCP/IP y tratar datos en hexadecimal.

\item Aprendí a actualizar en vivo la localización de un objeto y a implementarlo en el
entorno de Google Maps.

\item Aprendí el proceso de subir el producto final a las tiendas Play Store y App Store con
sus diferentes criterios, también el trato con personas asociadas al proyecto, así como
gestionar juntas, requisiciones y planes a futuro para la aplicación y llevar por mi propia
cuenta la dirección de un proyecto de software.

\end{enumerate}
\vfill % Intenta evitar el salto de página aquí
\subsection{Grupo Tomza (APSTA)}

\textbf{Puesto desempeñado:} 
\begin{itemize}
\item Desarrollador de Software.
\item Implementador de Software.
\item Lider de equipo de Software.

\end{itemize}

\textbf{Fechas:}
Desempeñé mis labores desde el 01 de febrero del 2023 hasta el 28 de febrero del 2024.


\textbf{Funciones:}
Desarrollar e implementar software requeridos para la misma empresa como módulos para
sistemas ERP o sistemas para gestionar las ventas del producto principal por otra parte el
sistema desarrollado también se implementó en la sucursal donde se nos indicó.



\textbf{Propuesta:}
\begin{enumerate}
\item Desarrollo de módulos que conforman el nuevo sistema ERP “ZAE” para el cambio de
plataforma que gestiona el área operativa dentro de la empresa para posteriormente
implementarla a nivel nacional.
\item Nuevo ambiente para la facturación de la empresa.
\item Desarrollo de sistema web para visualizar datos proporcionados del sistema ERP solo
para personal ejecutivo “ZAE EJECUTIVO”.

\end{enumerate}

\textbf{Implementación de propuestas}
\begin{enumerate}
\item Se desarrollaron módulos operativos para el nuevo sistema que gestiona a la empresa, se
trabajó con un framework basado en Python llamado Odoo para integrar los módulos al
nuevo sistema que reemplazará a la versión anterior, contiene una arquitectura MVC para
que la implementación de los módulos fuera más sencilla.


\item Se implementó y capacitó al personal para el nuevo sistema ERP en la región centro y
sur del país, anteriormente se recibió capacitación, para liderar equipos pequeños en la sede
que se encuentra en la ciudad de Puebla.


\item Se desarrolló un sistema web que visualiza datos de ventas y progresos dentro de la
empresa para personal gerencial y ejecutivo, estos datos son extraídos del sistema ERP
implementado anteriormente, se desarrolló con el Framework Angular y se trabajó en
conjunto con el administrador de la base de datos.

\item Se realizó un sistema conectado al ERP capaz de expedir facturas tomando en cuenta
los requerimientos necesarios por parte del SAT, cuenta con una arquitectura
Cliente/Servidor, realizado con el Framework Flutter y Odoo.


\end{enumerate}

\textbf{Impacto}
\begin{enumerate}
\item Actualmente la empresa usa este nuevo sistema para gestionar todas sus sedes y ventas,
tiene un gran impacto ya que organiza la materia prima de la empresa y fue de impacto
importante ya que es el sistema es el que administra a todas las sedes y sus respectivas
ventas.

\item Se implementó y capacito satisfactoriamente al personal de las diferentes sedes que se
indicaron dentro del plan del proyecto y actualmente usan este sistema sin problema alguno
tiene un impacto importante ya que es el sistema de ventas de la sede.

\item Actualmente el sistema ayuda a los gerentes a visualizar las ventas en las diferentes
sedes que cuenta la empresa, para así mejorar su producción o hacer ajuste al marco de
trabajo y administración que maneja, tiene un impacto importante ya que ayuda a la mejora
constante de la empresa.



\end{enumerate}

\textbf{Lo aprendido}
\begin{enumerate}
\item Aprendí a trabajar en colaboración con un equipo grande de desarrolladores cada uno
con funciones distintas o similares, apliqué procesos de calidad para testear software y
también a desarrollar software con el Framework Odoo y llevarlo a procesos de producción.


\item Desarrollé habilidades para capacitar personal en software nuevo, liderar y dirigir una
implementación, también utilizar métodos para agilizar la enseñanza en tiempos cortos, por
otra parte el manejo de personal gerencial para informar detalles de la implementación y
trabajo en conjunto.


\item Aprendí a manejar y a desarrollar software bajo demanda de un proyecto grande,
además de tratar directamente con el cliente del software y cumplir con las expectativas que
este tiene acerca del desarrollo.

\item Aprendí a desarrollar métodos de facturación con sus debidos requisitos legales que este
requiere para poder funcionar correctamente, testear respuestas por parte del timbrado legal
y creación de documentos en lenguaje Python y Flutter.

\end{enumerate}

\end{document}