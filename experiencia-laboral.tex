\documentclass[protocolo.tex]{subfiles}
\usepackage{cite}

\begin{document}
\newpage 
\section{Experiencia laboral}

En esta sección se describen las experiencias laborales de 
de en 3 empresas del sector tecnológico: 
Apia Ingeniería, Soluciones Móviles y Comunicaciones y Grupo Tomza. 
Para cada experiencia, se detallan las funciones desempeñadas, 
los proyectos en los que se participó, las tecnologías utilizadas y 
los aprendizajes adquiridos. Esta información permitirá comprender cómo Héctor Hugo Vidaña Arrieta aplicó sus conocimientos y habilidades en entornos laborales reales, contribuyendo al logro de los objetivos de cada empresa.

\subsection{Apia Ingeniería}

\textbf{Puesto desempeñado:} 

Las funciones desempeñadas incluyeron la administración de la base de datos que provee información a la empresa y a los sistemas desarrollados.
Los puestos desempeñados fueron:

\begin{itemize}
\item Desarrollador de software
\item Administrador de base de datos
\end{itemize}


\textbf{Proyectos:}
\begin{enumerate}
\item El sistema financiero \textit{DataFire} se diseñó para gestionar las ganancias,  
trabajadores,  proyectos  y  gastos  de  la  empresa.  
Permite  registrar  y  analizar  la  información  financiera  de  forma  centralizada,  lo  que  facilita  
la  toma  de  decisiones  sobre  las  inversiones  en  diferentes  proyectos.  
El  sistema  incluye  módulos  para  control  de  presupuestos,  gestión  de  nóminas,  análisis  de  rentabilidad,
planeador de proyectos y más.  Además,  permite  visualizar  la  información  en  gráficos  y  
reportes  que  facilitan  la  interpretación  de  los  datos  y  la  identificación  de  tendencias.\vspace{5mm} 

El proyecto \textit{DataFire} usa la arquitectura MVC para modularizar y separar la lógica de negocio de la interfaz y experiencia de usuario con el fin de modularizar y eficientizar la programación. 
También, se usó un \textit{Framework} que acelera el desarrollo de aplicaciones llamado \textit{Flutter} en conjunto con un modelo de datos relacional administrado haciendo uso de \textit{Postgres} 
y gestionando con \textit{NodeJs} para la creación de las diferentes API’s que el sistema necesitó.\vspace{5mm} 

Actualmente \textit{DataFire} es usado para gestionar proyectos, gastos y nominas. Solo está disponible en ordenadores Windows.

Se obtuvieron conocimientos en el desarrollo de algoritmos financieros como pagos de nómina, calculo de salarios, ISR, calculo de impuestos del seguro social, ingresos y egresos así también como la detección de tendencias como las ganancias y las perdidas, con el objetivo de optimizar la aplicación y generar datos más precisos. Además,  perfeccionó el manejo de la base de datos, implementando funciones como \textit{Stored Procedures} para optimizar la consulta de información.

\item El proyecto \textit{VehicleTracking} utiliza técnicas de Inteligencia Artificial para  contar  vehículos  a  partir  de  imágenes  de  video.  Se  implementaron  algoritmos  de visión  artificial,  aprendizaje  automático como \textit{DeepSort}, para  detectar  y  clasificar  los  vehículos  en  los  videos.  

Además  puede  procesar  videos  en  tiempo  real  o  de  forma  diferida.
\textit{VehicleTracking} es un sistema en desarrollo en lenguaje \textit{Python} que realiza aforos vehiculares por medio de video que usa modelos de detección como \textit{YOLOx8} y \textit{ultralytics} para la segmentación, seguimiento interpolado y visual para el conteo, aún se encuentra en desarrollo.\vspace{5mm} 


El sistema \textit{VehicleTracking} aún se encuentra en desarrollo debido a la complejidad del mismo pero se han realizado pruebas con aforos vehiculares no direccionales, lo cual presentó un visto bueno por parte de la empresa para seguir con el proyecto.


Se adquirió conocimiento de diferentes herramientas que se usan para el manejo de visión por computadora a través de(IA) como los son \textit{YOLO} que es el manejo de modelos de reconocimiento y \textit{Ultralytics} que se encarga de analizar y hacer las detecciones dentro del video o imagen, además de implementarlo en lenguaje \textit{Python}.\vspace{5mm} 
\end{enumerate}




\subsection{Soluciones Móviles y Comunicaciones}

Se realizaron los proyectos de software tales como el desarrollo completo de un
sistema, la mejora de un algoritmo en los sistemas que la empresa ya tiene y así mismo
manejar el trato directo con los asociados de los diferentes proyectos de software, 
manejo de juntas o nuevas requisiciones de software.
Los puestos desempeñados fueron:

\begin{itemize}
\item Desarrollador de software fullStack
\item Desarrollador de sistemas embebidos.
\end{itemize}



\textbf{Proyectos:}
\begin{enumerate}
\item El sistema CRM, basado en una conexión TCP/IP, utiliza dispositivos GPS llamados \textit{CalAmp} que se utilizan para obtener la ubicación en tiempo real de camiones y tráilers que transportan cargas importantes. La información de geolocalización y telemetría se transmite a través de la red y se almacena en una base de datos.

Además de la ubicación, el sistema registra datos como la velocidad, la temperatura, el estado del motor y otros parámetros relevantes para el seguimiento y control de las unidades. Esta información se visualiza en una interfaz web que permite a los usuarios monitorizar las rutas, el estado de las cargas y la condición de los vehículos en tiempo real.

Se desarrolló un programa en \textit{Python} llamado CRM que actúa como receptor de datos geoespaciales y telemétricos. Su funcionamiento se basa en los siguientes pasos:
\begin{enumerate}
    \item \textbf{Establecimiento de conexión:}  Inicia estableciendo una conexión segura mediante un protocolo de "handshake". Este proceso implica el intercambio de mensajes de confirmación (ACK), en formato hexadecimal, para asegurar la comunicación con el dispositivo emisor.
    \item \textbf{Recepción de información:} Una vez establecida la conexión,  recibe datos geoespaciales, como la ubicación geográfica (latitud, longitud, altitud), y telemétricos, que incluyen información del vehículo como velocidad, temperatura y nivel de combustible.
    \item \textbf{Almacenamiento en la nube:} Finalmente, envía la información recibida a un servicio de almacenamiento en la nube, como  \textit{AWS (Amazon Web Services)}, \textit{Google Cloud Platform} o \textit{Microsoft Azure}, para su posterior análisis y procesamiento.
\end{enumerate}


Actualmente CRM es utilizado principalmente para alimentar a la base de datos de
ubicaciones para actualizar el mapa que se encuentra de manera online a los que cuentan
con el servicio de localización y telemetría y tiene un impacto importante ya que es el
sistema principal de la empresa.


Se adquirió experiencia para mejorar las conexiones TCP/IP y tratar datos en hexadecimal.


\item El proyecto \textit{FollowMe}, basado en la tecnología \textit{CalAmp}, permite el rastreo de paquetes y cargas dentro de un área específica. Los paquetes, etiquetados con dispositivos de seguimiento, transmiten su ubicación en tiempo real. Esta información se visualiza en un mapa digital,  ofreciendo a los usuarios la posibilidad de conocer la posición exacta de cada paquete y su estado de entrega. Adicionalmente, el sistema genera reportes y estadísticas sobre el movimiento de las cargas, facilitando la gestión logística y la optimización de las rutas de entrega.

Se desarrolló una interfaz que visualiza paquetes dentro del rango de ubicación del usuario llamado \textit{FollowMe}. Esta interfaz, basada en la tecnología de \textit{Google Maps} y \textit{Flutter}, utiliza una arquitectura cliente/servidor para mostrar la información recibida por el oyente previamente desarrollado llamado CRM.


El proyecto \textit{FollowMe} se encuentra en uso personal por parte de la empresa y ofrece sus servicios a
quienes tienen sistemas \textit{CalAmp} previamente instalados en trailers o camiones y tiene un
impacto medio ya que no todos pueden usar este servicio.

Se obtuvo conocimiento para actualizar en vivo la localización de un objeto e implementarlo en el
entorno de \textit{Google Maps}.



\item El proyecto \textit{AdCom} ofrece una amplia gama de funcionalidades para la  administración  de  fraccionamientos  en  el  estado  de  Chihuahua.  Los  residentes  pueden  acceder  al  sistema  a  través  de  una  aplicación  móvil  o  una  plataforma  web  para:

\begin{itemize}
\item Abrir la puerta principal de forma remota.
\item Apartar amenidades como canchas deportivas o salones de eventos.
\item Consultar y pagar sus deudas de mantenimiento, servicios o cuotas.
\item Reportar fallas o incidentes dentro del fraccionamiento.
\item Recibir notificaciones sobre eventos o comunicados importantes.
\item Comunicarse con la administración del fraccionamiento.
\end{itemize}

\textit{AdCom} que administra fraccionamientos en el estado de Chihuahua,
Chihuahua haciendo uso de \textit{Clean Architecture} para desarrollos de larga duración y modulares
dentro del \textit{Framework} de \textit{Flutter} con el fin de que más programadores pudieran
implementar más módulos y mejorar el entendimiento de código. Se usó un diseño propio
para la aplicación que prioriza la originalidad y usabilidad para los diferentes usuarios y además hubo
trato directo con los asociados del proyecto para retroalimentación o más requisiciones.


Actualmente la aplicación \textit{AdCom} se encuentra en uso y se puede encontrar en tiendas \textit{Play
Store} y \textit{App Store}, tiene buen funcionamiento e impacto en las comunidades de
fraccionamientos en Chihuahua.

Se mejoró en el proceso de subir el producto final a las tiendas \textit{Play Store} y \textit{App Store} con
sus diferentes criterios, también se mejoró el trato con personas asociadas al proyecto, así como
gestionar juntas, requisiciones y planes a futuro para la aplicación y llevar por cuenta propia la dirección de un proyecto de software.

\end{enumerate}


 % Intenta evitar el salto de página aquí
\subsection{Grupo Tomza}

Se realizaron los proyectos de software requeridos por la empresa tales como módulos para
sistemas ERP o sistemas para gestionar las ventas del producto principal.\vspace{5mm}

Los puestos desempeñados fueron:

\begin{itemize}
\item Desarrollador de software.
\item Implementador de software.
\item Líder de equipo de software.
\end{itemize}

\textbf{Proyectos:}
\begin{enumerate}
\item ZAE fue un proyecto ERP,  enfocados  en  la  gestión  del  área  operativa  de  la  empresa.  Donde se  incluyen  funcionalidades  para  venta de gas medido, la gestión de órdenes de trabajo, el control de la producción, el seguimiento de la distribución y reportes informativos.  El  objetivo  principal  de  este  desarrollo  fue  optimizar  los  procesos  operativos,  mejorar  la  eficiencia  y  facilitar  la  toma  de  decisiones  en  el  área  operativa.


Se desarrollaron módulos operativos para el nuevo sistema que gestiona a la empresa llamado ZAE, se
trabajó con un \textit{Framework} basado en \textit{Python} llamado \textit{Odoo} para integrar los módulos al
nuevo sistema que reemplazará a la versión anterior, contiene una arquitectura modelo vista controlador (MVC) para
que la implementación de los módulos fuera más sencilla.


Actualmente la empresa usa este nuevo sistema para gestionar todas sus sedes y ventas,
tiene un gran impacto ya que organiza la materia prima de la empresa y fue de impacto
importante ya que es el sistema el que administra a todas las sedes y sus respectivas
ventas.

Se adquirieron conocimientos para trabajar en colaboración con un equipo grande de desarrolladores cada uno
con funciones distintas o similares, tambien se aplicaron procesos de calidad para testear software y
también a desarrollar software con el \textit{Framework} \textit{Odoo} y llevarlo a procesos de producción.


\item El proyecto ZAE EJECUTIVO es un sistema  de  \textit{business intelligence}  para  el  personal  ejecutivo  de  la  empresa.  Este  sistema  permite  visualizar  datos  clave  del  sistema  ERP  "ZAE"  en  un  formato  accesible  y  fácil  de  entender,  con  el  objetivo  de  facilitar  la  toma  de  decisiones  estratégicas.  "ZAE  EJECUTIVO"  ofrece \textit{Dashboards} interactivos, reportes personalizables, gráficos dinámicos que  permiten  a  los  ejecutivos  analizar  el  rendimiento  de  la  empresa,  identificar  tendencias  y  áreas  de  oportunidad.


Se desarrolló un sistema web que se llama ZAE Ejecutivo que visualiza datos de ventas y progresos dentro de la
empresa para personal gerencial y ejecutivo, estos datos son extraídos del sistema ERP
implementado anteriormente, se desarrolló con el \textit{Framework} Angular y se trabajó en
conjunto con el administrador de la base de datos.


Actualmente el sistema ayuda a los gerentes a visualizar las ventas en las diferentes
sedes que cuenta la empresa, para así mejorar su producción o hacer ajuste al marco de
trabajo y administración que maneja, tiene un impacto importante ya que ayuda a la mejora
constante de la empresa.


Se adquirió experiencia en el manejo y desarrollo de software a gran escala, específicamente en la creación de un sistema para la gestión de una empresa a nivel nacional. Este proceso implicó la interacción directa con el cliente, comprendiendo y satisfaciendo sus expectativas respecto al desarrollo del software.


\end{enumerate}


\end{document}