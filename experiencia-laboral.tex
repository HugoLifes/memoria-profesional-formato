\documentclass[protocolo.tex]{subfiles}
\usepackage{cite}

\begin{document}
\newpage 
\section{Experiencia laboral}

En esta sección se describen las experiencias laborales del Ing. 
Héctor Hugo Vidaña Arrieta en 3 empresas del sector tecnológico: 
Apia ingeniería, soluciones moviles y comunicacions y grupo tomza. 
Para cada experiencia, se detallan las funciones desempeñadas, 
los proyectos en los que se participó, las tecnologías utilizadas y 
los aprendizajes adquiridos. Esta información permitirá comprender cómo el 
Ing. Héctor Hugo Vidaña Arrieta aplicó sus conocimientos y habilidades en entornos laborales reales, contribuyendo al logro de los objetivos de cada empresa.

\subsection{Apia ingeniería}

\textbf{Acerca de:} 
APIA Ingeniería es un despacho donde se llevan a cabo proyectos de infraestructura del transporte para las diversas áreas de planeación, proyecto ejecutivo, construcción, operación y mantenimiento de los sistemas carretero, ferroviario y aeroportuario.

\textbf{Puesto desempeñado:} 
\begin{itemize}
\item Desarrollador de software
\item Administrador de base de datos
\end{itemize}

\textbf{Fechas:}
Desempeñó labores desde 3 de febrero de 2018 hasta el 20 de noviembre del 2019 para después volver a realizar labores el 25 de marzo del 2024.

\textbf{Funciones:}
El Ing. Vidaña fue responsable de realizar y administrar los proyectos de software de la empresa. Sus funciones incluyeron la administración de la base de datos que provee información a la empresa y a los sistemas desarrollados.

\textbf{Propuestas:}
\begin{enumerate}
\item Este sistema financiero se diseñó para gestionar las ganancias,  trabajadores,  proyectos  y  gastos  de  la  empresa.  Permite  registrar  y  analizar  la  información  financiera  de  forma  centralizada,  lo  que  facilita  la  toma  de  decisiones  sobre  las  inversiones  en  diferentes  proyectos.  El  sistema  incluye  módulos  para  control  de  presupuestos,  gestión  de  nóminas,  análisis  de  rentabilidad, planeador de proyectos y más.  Además,  permite  visualizar  la  información  en  gráficos  y  reportes  que  facilitan  la  interpretación  de  los  datos  y  la  identificación  de  tendencias.
\item "Este sistema utiliza técnicas de Inteligencia Artificial (IA) para  contar  vehículos  a  partir  de  imágenes  de  video.  Se  implementaron  algoritmos  de visión  artificial,  aprendizaje  automático, para  detectar  y  clasificar  los  vehículos  en  los  videos.  El  sistema  puede  procesar  videos  en  tiempo  real  o  de  forma  diferida,  y  genera  reportes  con  el  conteo  de  vehículos  por  tipo,  hora,  día,  etc.

\begin{itemize}
    \item La Inteligencia Artificial (IA)  es  la  simulación  de  procesos  de  inteligencia  humana  por  parte  de  máquinas,  especialmente  sistemas  informáticos.  Estos  procesos  incluyen  el  aprendizaje  (la  adquisición  de  información  y  reglas  para  el  uso  de  la  información),  el  razonamiento  (el  uso  de  las  reglas  para  llegar  a  conclusiones  aproximadas  o  definitivas)  y  la  autocorrección. \cite{russell2010}
\end{itemize}

\end{enumerate}

\textbf{Implementación de propuestas}
\begin{enumerate}
\item Se usó la arquitectura MVC para modularizar y separar la lógica de negocio de la interfaz y experiencia de usuario con el fin de modularizar y eficientizar la programación, también, se usó un framework que acelera el desarrollo de aplicaciones llamado flutter en conjunto de un modelo de datos relacional administrado haciendo uso de Postgres y gestionando con NodeJs para la creación de las diferentes API’s que el sistema necesita.
\item Sistema en desarrollo en lenguaje Python que realiza aforos vehiculares por medio de video que usa modelos de detección como YOLOx8 para la detección y \textit{ultralytics} para la segmentación, seguimiento interpolado y el seguimiento visual para el conteo, aún se encuentra en desarrollo.
\end{enumerate}

\textbf{Impacto}
\begin{enumerate}
\item Actualmente el sistema es usado para gestionar proyectos y gastos y tiene un buen funcionamiento, actualmente solo está disponible en ordenadores Windows y se está trabajando para que llegue a más plataformas.
\item El sistema aún se encuentra en desarrollo debido a la complejidad del mismo pero se han realizado pruebas con aforos vehiculares no direccionales, lo cual presentó un visto bueno por parte de la empresa para seguir con el proyecto.
\end{enumerate}

\textbf{Lo aprendido}
\begin{enumerate}
\item El Ing. Vidaña dominó en el ambito de los algoritmos financieros y detecciones de tendencia con el fin de mejorar el impacto de la aplicación a la hora de generar datos más verídicos, por otra parte mejoré en el manejo de la base de datos y realizar funciones como Stored Procedures para optimizar la llamada de datos.
\end{enumerate}

\vfill % Intenta evitar el salto de página aquí 
\subsection{Soluciones móviles y comunicaciones S.A de C.V}

\textbf{Misión:}

Somos un aliado importante de nuestros clientes, generándoles ahorro y riqueza en sus organizaciones, a través del desarrollo de soluciones tecnológicas que ofrecen información clara, veraz y oportuna.


\textbf{Vision:}

Ser un organización integral, solida y líder nacional en innovación y en el uso de tecnológicas para el desarrollo de soluciones, que generen beneficio económico a nuestros clientes; apoyados siempre en un equipo humano competente, comprometido y auto-motivado.


\textbf{Puesto desempeñado:} 
\begin{itemize}
\item Desarrollador de software fullStack
\item Desarrollador de sistemas embebidos.
\end{itemize}

\textbf{Fechas:}
Desempeño labores desde el 18 de marzo del 2021 hasta el 26 de febrero del 2022.

\textbf{Funciones:}
Realizar los proyectos de software que la empresa requirió, como el desarrollo completo de
un sistema o la mejora de un algoritmo en sistemas que la empresa ya tiene y así mismo
manejar el trato directo con los asociados de los diferentes proyectos de software que se
están realizando, como manejo de juntas o nuevas requisiciones del software.


\textbf{Propuestas:}
\begin{enumerate}
\item Este sistema se basa en una conexión TCP/IP  y utiliza dispositivos Calamp para  obtener la ubicación de camiones y tráileres con cargas importantes en tiempo real. La información de geolocalización y telemetría se transmite a través de la red y se almacena en una base de datos.  Además de la ubicación,  el sistema  registra  datos  como la velocidad, la temperatura, el estado del motor y otros parámetros relevantes para el  seguimiento  y  control  de  las  unidades.  Esta  información  se  visualiza  en  una  interfaz  web  que  permite  a  los  usuarios  monitorear  las  rutas,  el  estado  de  las  cargas  y  la  condición  de  los  vehículos  en  tiempo  real.
\item Este sistema utiliza la tecnología Calamp para  rastrear  paquetes  y  cargas  dentro  de  una  zona  determinada.  Los  paquetes  se  etiquetan  con  dispositivos  de  seguimiento  que  transmiten  su  ubicación  en  tiempo  real.  La  información  se  visualiza  en  un  mapa  digital  que  permite  a  los  usuarios  conocer  la  posición  exacta  de  cada  paquete  y  su  estado  de  entrega.  El  sistema  también  genera  reportes  y  estadísticas  sobre  el  movimiento  de  las  cargas,  lo  que  facilita  la  gestión  logística  y  la  optimización  de  las  rutas  de  entrega.
\item Este sistema ofrece una amplia gama de funcionalidades para la  administración  de  fraccionamientos  en  el  estado  de  Chihuahua.  Los  residentes  pueden  acceder  al  sistema  a  través  de  una  aplicación  móvil  o  una  plataforma  web  para:

\begin{itemize}
\item Abrir la puerta principal de forma remota.
\item Apartar amenidades como canchas deportivas o salones de eventos.
\item Consultar y pagar sus deudas de mantenimiento, servicios o cuotas.
\item Reportar fallas o incidentes dentro del fraccionamiento.
\item Recibir notificaciones sobre eventos o comunicados importantes.
\item Comunicarse con la administración del fraccionamiento.
\end{itemize}
\end{enumerate}

\textbf{Implementación de propuestas}
\begin{enumerate}
\item Se desarrolló un oyente en lenguaje Python que recibe un ACK en hexadecimal como
chunk data y hacer un Handshake para responder un ACK para después recibir información
geoespacial y telemétrica para después subir la información a la nube

\item Se desarrolló una interfaz que visualiza paquetes dentro del rango en el que te
encuentras a partir del oyente realizado para recibir información usando tecnología de
\textit{Google Maps} y \textit{flutter} usando una arquitectura Cliente/Servidor.

\item Se desarrolló un sistema que administra fraccionamientos en el estado de Chihuahua,
Chih haciendo uso de \textit{clean architecture} para desarrollos de larga duración y modulares
dentro del Framework de \textit{flutter} con el fin de que más programadores pudieran
implementar más módulos y mejorar el entendimiento de código, se usó un diseño propio
para la aplicación que prioriza la originalidad y usabilidad para los diferentes usuarios y
trato directo con los asociados del proyecto para retroalimentación o más requisiciones
\end{enumerate}

\textbf{Impacto}
\begin{enumerate}
\item Actualmente el oyente es utilizado principalmente para alimentar a la base de datos de
ubicaciones para actualizar el mapa que se encuentra de manera online a los que cuentan
con el servicio de Localización y Telemetría tiene un impacto importante ya que es el
algoritmo principal de la empresa.

\item El sistema se encuentra en uso personal por parte de la empresa y ofrece sus servicios a
quienes tienen sistemas \textit{Calamp} previamente instalados en trailers o camiones, tiene un
impacto medio ya que no todos pueden usar este servicio.

\item Actualmente la aplicación se encuentra en uso y se puede encontrar en tiendas \textit{play
store} y \textit{app store}, tiene buen funcionamiento e impacto en las comunidades de
fraccionamientos en Chihuahua.

\end{enumerate}

\textbf{Lo aprendido}
\begin{enumerate}
\item Adquirió experiencia para mejorar las conexiones TCP/IP y tratar datos en hexadecimal.

\item Se obtuvo conocimiento para actualizar en vivo la localización de un objeto y a implementarlo en el
entorno de Google Maps.

\item Se mejoró en el proceso de subir el producto final a las tiendas \textit{play store} y \textit{app store} con
sus diferentes criterios, también el trato con personas asociadas al proyecto, así como
gestionar juntas, requisiciones y planes a futuro para la aplicación y llevar por mi propia
cuenta la dirección de un proyecto de software.

\end{enumerate}
\vfill % Intenta evitar el salto de página aquí
\subsection{Grupo tomza (APSTA)}

\textbf{Misión:}

Grupo tomza es una empresa comprometida con abastecer, proveer y satisfacer permanentemente las necesidades energéticas de nuestros clientes en el suministro de Gas L.P. cumpliendo las necesidades y expectativas de manera segura con eficiencia y eficacia, brindando al mercado doméstico, industrial y comercial, un servicio amable, continuo, exacto y de calidad, con el compromiso y esfuerzo del recurso humano, ofreciendo políticas de calidad con las mejores condiciones de seguridad y protección al medio ambiente

\textbf{Visión:} 

Ser el grupo gasero líder en el mercado nacional y centroamericano en la importación, almacenamiento, transportación, distribución y venta de Gas LP, con talento humano altamente capacitado y socialmente responsable, con una participación creciente en el mercado internacional, afianzarnos con rentabilidad en el sector industrial, doméstico y comercial mediante altos estándares de seguridad orientando nuestros esfuerzos hacia una mejora continua para lograr el liderazgo y el crecimiento de la corporación.

\textbf{Puesto desempeñado:} 
\begin{itemize}
\item Desarrollador de software.
\item Implementador de software.
\item Líder de equipo de software.

\end{itemize}

\textbf{Fechas:}
Desempeñó labores desde el 01 de febrero del 2023 hasta el 28 de febrero del 2024.


\textbf{Funciones:}
El Ing. Vidaña desarrolló e implementó software requeridos para la empresa como módulos para
sistemas ERP o sistemas para gestionar las ventas del producto principal por otra parte el
sistema desarrollado también se implementó en la sucursal donde se nos indicó.


\textbf{Propuestas:}
\begin{enumerate}
\item En este proyecto,  se  desarrollaron  módulos  específicos  para  el  sistema  ERP  "ZAE",  enfocados  en  la  gestión  del  área  operativa  de  la  empresa.  Estos  módulos  incluyen  funcionalidades  para  venta de gas medido, la gestión de órdenes de trabajo, el control de la producción, el seguimiento de la distribución.  El  objetivo  principal  de  este  desarrollo  fue  optimizar  los  procesos  operativos,  mejorar  la  eficiencia  y  facilitar  la  toma  de  decisiones  en  el  área  operativa..
\item Este proyecto se centró en la creación de un nuevo ambiente para la facturación de la empresa,  con el objetivo de emitir facturas fuera del entorno ZAE para personas morales o clientes que requerian de factura rapida y fue realizado en flutter y odoo ademas  se  integró  con  el  sistema  ERP  "ZAE"  para  garantizar  la  consistencia  de  la  información.
\item El sistema web "ZAE EJECUTIVO" se  desarrolló  como  una  herramienta  de  \textit{business intelligence}  para  el  personal  ejecutivo  de  la  empresa.  Este  sistema  permite  visualizar  datos  clave  del  sistema  ERP  "ZAE"  en  un  formato  accesible  y  fácil  de  entender,  con  el  objetivo  de  facilitar  la  toma  de  decisiones  estratégicas.  "ZAE  EJECUTIVO"  ofrece dashboards interactivos, reportes personalizables, gráficos dinámicos que  permiten  a  los  ejecutivos  analizar  el  rendimiento  de  la  empresa,  identificar  tendencias  y  áreas  de  oportunidad.

\begin{itemize}
    \item Business Intelligence (BI) engloba un conjunto de metodologías, procesos, arquitecturas y tecnologías que transforman datos sin procesar en información significativa   
     que impulsa acciones empresariales rentables. BI puede incluir la recopilación de datos, el análisis de datos, el almacenamiento de datos, la minería de datos, la generación de informes, la consulta, el análisis estadístico, la predicción y la minería de texto. \cite{turban2010}
\end{itemize}
    

\end{enumerate}

\textbf{Implementación de propuestas}
\begin{enumerate}
\item Se desarrollaron módulos operativos para el nuevo sistema que gestiona a la empresa, se
trabajó con un framework basado en Python llamado Odoo para integrar los módulos al
nuevo sistema que reemplazará a la versión anterior, contiene una arquitectura modelo vista controlador (MVC) para
que la implementación de los módulos fuera más sencilla.


\item Se implementó y capacitó al personal para el nuevo sistema de planificacion de recursos empresariales (ERP) en la región centro y
sur del país, anteriormente se recibió capacitación, para liderar equipos pequeños en la sede
que se encuentra en la ciudad de Puebla.


\item Se desarrolló un sistema web que visualiza datos de ventas y progresos dentro de la
empresa para personal gerencial y ejecutivo, estos datos son extraídos del sistema ERP
implementado anteriormente, se desarrolló con el \textit{framework} angular y se trabajó en
conjunto con el administrador de la base de datos.

\item Se realizó un sistema conectado al ERP capaz de expedir facturas tomando en cuenta
los requerimientos necesarios por parte del SAT, cuenta con una arquitectura
Cliente/Servidor, realizado con el \textit{framework} flutter y odoo.


\end{enumerate}

\textbf{Impacto}
\begin{enumerate}
\item Actualmente la empresa usa este nuevo sistema para gestionar todas sus sedes y ventas,
tiene un gran impacto ya que organiza la materia prima de la empresa y fue de impacto
importante ya que es el sistema es el que administra a todas las sedes y sus respectivas
ventas.

\item Se implementó y capacito satisfactoriamente al personal de las diferentes sedes que se
indicaron dentro del plan del proyecto y actualmente usan este sistema sin problema alguno
tiene un impacto importante ya que es el sistema de ventas de la sede.

\item Actualmente el sistema ayuda a los gerentes a visualizar las ventas en las diferentes
sedes que cuenta la empresa, para así mejorar su producción o hacer ajuste al marco de
trabajo y administración que maneja, tiene un impacto importante ya que ayuda a la mejora
constante de la empresa.



\end{enumerate}

\textbf{Lo aprendido}
\begin{enumerate}
\item Adquirió conocimientos para trabajar en colaboración con un equipo grande de desarrolladores cada uno
con funciones distintas o similares, apliqué procesos de calidad para testear software y
también a desarrollar software con el \textit{framework} odoo y llevarlo a procesos de producción.


\item Desarrolló habilidades para capacitar personal en software nuevo, liderar y dirigir una
implementación, también utilizar métodos para agilizar la enseñanza en tiempos cortos, por
otra parte el manejo de personal gerencial para informar detalles de la implementación y
trabajo en conjunto.


\item Se obtuvo conocimiento para manejar y desarrollar software bajo demanda de un proyecto grande,
además de tratar directamente con el cliente del software y cumplir con las expectativas que
este tiene acerca del desarrollo.

\item El Ing. Vidaña se familiarizo a desarrollar métodos de facturación con sus debidos requisitos legales que este
requiere para poder funcionar correctamente, testear respuestas por parte del timbrado legal
y creación de documentos en lenguaje \textit{python} y \textit{flutter}.

\end{enumerate}

\end{document}